\documentclass{beamer}
\usepackage[latin1]{inputenc}

\mode<presentation>{ \usetheme{Warsaw} } 

\title{Luma - Lpeg-based Lua macros}

\author{Fabio Mascarenhas}

\begin{document}

\begin{frame}
\titlepage
\end{frame}

\begin{frame}
\frametitle{Motivation}

\begin{itemize}
\item Macros that are not bound by Lua's syntax...
\item ...but clearly delimited when mixed with Lua code
\item Defining a macro's grammar and the code it generates
should be as easy as possible
\item Lpeg's re.lua for grammar and Cosmo for code templates
\item Inspired by Scheme's \emph{define-syntax}
\end{itemize}

\end{frame}

\begin{frame}[fragile]
\frametitle{A Luma Macro}

\begin{verbatim}
#!/usr/bin/env luma

require_for_syntax[[automaton]]

local aut = automaton [[
  init: c -> more
  more: a -> more
        d -> more
        ' ' -> more
        r -> finish
  finish: accept
]]

print(aut("cadar"))
print(aut("cad ddar"))
print(aut("caxadr"))

\end{verbatim}

\end{frame}

\begin{frame}[fragile]
\frametitle{Using Luma Macros}

\begin{itemize}
\item Macro application: \emph{selector} \verb|[[| \emph{data} \verb|]]|
\item \emph{data} is any string that contains balanced \verb|[[|s and \verb|]]|s
\item It's up to the macro to parse and interpret it
\item Expansion is repeated until there are no more macros to expand, so
macros can expand to macros
\item Command line script, \emph{luma}, expands macros in files before running them
\item \emph{require\_ for\_ syntax} macro require's a library in expansion time, so any
macros defined by the library are available to the code that follows it
\item Similarity to Converge's \emph{DSL blocks}
\end{itemize}

\end{frame}

\begin{frame}[fragile]
\frametitle{Defining \emph{automaton} - Syntax}
A Luma macro needs a \emph{syntax}, a set of extra \emph{definitions and builders}, and
a \emph{code template}. The syntax for \emph{automaton}:
\begin{verbatim}
local syntax = [[
  aut <- _ state+ -> build_aut
  char <- ([']{[ ]}['] / {.}) _
  rule <- (char '->' _ {name} _) -> build_rule 
  state <- ( {name} _ ':' _ rule* -> {} {'accept'?} _ ) 
    -> build_state
]]
\end{verbatim}
Luma uses re.lua, and defines a few default rules that can be used by
any macro: spaces and Lua comments (\emph{\_}), Lua names (\emph{name}),
Lua numbers (\emph{numbers}), and Lua strings (\emph{string}, \emph{shortstring},
and \emph{longstring}).
\end{frame}

\begin{frame}[fragile]
\frametitle{Defining \emph{automaton} - Definitions}
\emph{automaton}'s definitions are the capture functions, they will build a description
of the automaton suitable for use by the code template.
\begin{verbatim}
local defs = {
  build_rule = function (c, n)
    return { char = c, next = n }
  end,
  build_state = function (n, rs, accept)
    local final = tostring(accept == 'accept')
    return { name = n, rules = rs, final = final }
  end,
  build_aut = function (...)
    return { init = (...).name, states = { ... }, 
      substr = luma.gensym(), c = luma.gensym(),
      input = luma.gensym(), rest = luma.gensym() }
  end
}
\end{verbatim}
\end{frame}

\begin{frame}[fragile]
\frametitle{Defining \emph{automaton} - Code Template}
Luma uses \emph{Cosmo} for its code templates, Cosmo templates expect a table (the first
capture the grammar returns on match) and use this table to fill the template.
\begin{verbatim}
local code = [[
  (function ($input)
    local $substr = string.sub
    $states[=[
      local $name
    ]=]
    $states[=[
      $name = function ($rest)
        if #$rest == 0 then
          return $final
        end
\end{verbatim}
\emph{continues...}
\end{frame}

\begin{frame}[fragile]
\frametitle{Defining \emph{automaton} - Code Template contd.}
\begin{verbatim}
        local $c = $substr($rest, 1, 1)
        $rest = $substr($rest, 2, #$rest)
        $rules[==[
          if $c == '$char' then
            return $next($rest)
          end
        ]==]
        return false
      end
    ]=]
    return $init($input)
  end)]]
\end{verbatim}
Macro hygiene is currently the macro programmer's responsibility.
\end{frame}

\begin{frame}[fragile]
\frametitle{Using Lua's syntax}

Defining macros that use a subset of Lua's syntax is possible, just use \emph{Leg}'s Lua parser. If
Luma detects that Leg is present it modifies the parser so Luma's macro applications are legal Lua syntax.

Example (list comprehensions):

\begin{verbatim}
x = L[[i | i <- 1,3]] -- x = {1, 2, 3}
y = L[[L[[j | j <- 1, 3]] | i <- 1, 3]] 
    -- y = {{1, 2, 3}, {1, 2, 3}, {1, 2, 3}}
z = L[[tostring(automaton[[init: c -> more 
                           more: a -> more
                                 d -> more
                                 r -> finish
                           finish: accept]](line))
       | line <- io.lines()]]
    -- tries to recognize each line from stdin with
    -- the automaton and collect results in z
\end{verbatim}

\end{frame}

\begin{frame}[fragile]
\frametitle{List comprehensions - Syntax}
The list comprehension macro imports the \emph{Exp} rule from Leg as \emph{exp}:

\begin{verbatim}
local syntax = [[
  comp <- (_ {exp} _ '|'  _ (numfor / genfor) _)
    -> build_comp
  numfor <-  ({name} _ {~ '<-' -> '=' ~} _ 
    {exp _ ',' _ exp _ (',' _ exp)?}) -> concat
  genfor <- ({name} _ (',' _ name)* _ {~ '<-' -> 'in' ~}
    _ {exp _ (',' _ exp)*}) -> concat
]]
\end{verbatim}

The builders are boilerplate, \emph{concat} concatenates all captures
separated by whitespace, \emph{build\_ comp} creates a table with the expression,
the \emph{for} part and a gensym for the final list.
\end{frame}

\begin{frame}[fragile]
\frametitle{List comprehensions - Code}
The \emph{expression} and \emph{for} parts are already in the correct Lua syntax, so the
code is straightforward:

\begin{verbatim}
local code = [[
  (function ()
    local $list = {}
    for $for_part do
      $list[#$list + 1] = $exp
    end
    return $list
  end)()
]]
\end{verbatim}
\end{frame}

\begin{frame}[fragile]
\frametitle{Inlining definitions}
Luma defines a \emph{meta} macro that lets you execute arbitray code
at expansion time. This lets you define macros in the same file they
are used:

\begin{verbatim}
meta[[
  local function fact(n)
    n = tonumber(n)
    if n<=1 then return 1 else return n*fact(n-1) end
  end
  luma.define("fact", "{number}", fact) 
]]

print(fact[[3]]) -- expands to print(6)
\end{verbatim}

If a macro is very simple you can use a function that returns
its expansion instead of a Cosmo template.
\end{frame}

\begin{frame}[fragile]
\frametitle{Final Remarks and Issues}
\begin{itemize}
\item Currently Luma has several other examples: class definitions,
try/catch/finally (with proper treatment of return, inc, 
match/with using Lpeg, Python-like \emph{from module import symbol}, and others
\item Most macros defined in less than 50 lines of code, the simplest
in less than 10 lines, including templates
\item But no good error reporting yet (just a generic parse error if
a macro doesn't match)
\item gensym doesn't solve the problem of user code shadowing identifiers the macro needs
\end{itemize}
\end{frame}

\begin{frame}
\begin{center}
Questions? Suggestions? Flames? :-)
\end{center}
\end{frame}

\end{document}
